\documentclass{article}

%% Font related
\renewcommand{\familydefault}{\sfdefault}
\usepackage[T1]{fontenc}
\usepackage{courier}
%\renewcommand*\familydefault{\ttdefault} %% Only if the base font of the document is to be typewriter style

%\usepackage[zerostyle=d]{newtxtt} %% Various versions of zeros available. See documentation for details

%% Document related
\usepackage[utf8]{inputenc}
\usepackage[english]{babel}
\usepackage[margin=0.5in]{geometry}
\usepackage{flushend}

%% Maths related 
\usepackage{mathtools}

%% Figure related packages
\usepackage{graphicx}
\usepackage{float}

%% Table related
\usepackage{tabularx}
\usepackage[table]{xcolor}
\definecolor{lightgray1}{gray}{0.86}
\definecolor{lightgray2}{gray}{0.9}
\definecolor{lightgray3}{gray}{0.5}
\usepackage{multirow}

%% Multi-columns related
\usepackage{multicol}
\setlength{\columnsep}{1cm}

%% Document Sections related
\usepackage{titlesec}
\usepackage{tikz}
\usetikzlibrary{shapes.misc}
\newcommand\titlebar{%
\tikz[baseline,trim left=3.1cm,trim right=3cm] {
    \fill [cyan!25] (2.5cm,-1ex) rectangle (\textwidth-\textwidth/3, 2.5ex);
    \node [
        fill=cyan!60!white,
        anchor= base east,
        rounded rectangle,
        minimum height=3.5ex] at (3cm,0) {
        \textbf{\thesection.}
    };
}%
}
\titleformat{\section}{\large}{\titlebar}{0.1cm}{}
\renewcommand*{\thesection}{\arabic{section}}


%% Document title related
\title{
  \textbf{R Cheat sheet}
}
\author{by Paulo Monteiro}
\date{}

%%%%%%%%%%%%%%%%%%%%%%%%%%%%%%%%%%%%%%
%% BEGIN DOCUMENT
%%%%%%%%%%%%%%%%%%%%%%%%%%%%%%%%%%%%%%
\begin{document}
  \begin{multicols}{2}
  \maketitle

  \section{\textbf{Getting Started}}
  \rowcolors{1}{lightgray1}{lightgray2}
  \begin{tabularx}{\textwidth/2}{X|l}
    \texttt{\textbf{ x = 16; y $<-$"a string"}}\\
    \color{lightgray3}Assigning a value to an object/variable. R is case sensitive\\
    \hline
    
    \texttt{\textbf{ls(); objects()}}\\
    \color{lightgray3}Lists all the objects created in the workspace memory\\
    \hline
    
    \texttt{\textbf{rm(y)}}\\
    \color{lightgray3}Removes an object from the workspace memory\\
    \hline
    
    \texttt{\textbf{exp(x); sqrt(16); log(x); abs(x)}}\\
    \color{lightgray3}A few math operations allowed\\
    \hline
  \end{tabularx}

  \section{\textbf{Vectors, Lists and Matrices}}
  \rowcolors{1}{lightgray1}{lightgray2}
  \begin{tabularx}{\textwidth/2}{X|l}
    \texttt{\textbf{x1=c(1,2,3,4,5); colours=c("blue", "red")}}\\
    \color{lightgray3}Creating a vector of integers and strings\\
    \hline
    
    \texttt{\textbf{2:9; seq(from=2, to=10, by=1)}}\\
    \color{lightgray3}Creating a sequence of values\\
    \hline
    
    \texttt{\textbf{rep(1,times=5);rep("hello",times=5)}}\\
    \color{lightgray3}Repeat a number '1' or string 'hello' 5 times\\
    \hline
    
    \texttt{\textbf{x=1:5; y=x+10}}\\
    \color{lightgray3}Creating a list and adding 10 to each element (*, /, - and + apply)\\
    \hline
    
    \texttt{\textbf{x=1:5; y=5:1; x+y}}\\
    \color{lightgray3}if two vectors are of the same length we may add/subtract/mult/div its elements\\
    \hline
    
    \texttt{\textbf{x=1:5; x[1]=16}}\\
    \color{lightgray3}Assigning 16 to index '1' of vector x\\
    \hline
    
    \texttt{\textbf{y[c(1,3)]}}\\
    \color{lightgray3}Select the first two elements of vector y\\
    \hline
    
    \texttt{\textbf{y[-c(1,3)]}}\\
    \color{lightgray3}Select all elements of vector y apart from the two first elements\\
    \hline
    
    \texttt{\textbf{y[y<3]}}\\
    \color{lightgray3}Select all elements less than 3 from vector y\\
    \hline 
  \end{tabularx}
  

  \begin{tabularx}{\textwidth/2}{X|l}  
    \texttt{\textbf{mat=matrix(c(1,2,3,4,5,6,7,8,9), nrow=3, byrow=TRUE)}}\\
    \color{lightgray3}Create a matrix 3x3 with values entered by row\\
    \hline
     
    \texttt{\textbf{mat[1,2]}}\\
    \color{lightgray3}Extract an element from the matrix\\
    \hline
    
    \texttt{\textbf{mat[c(1,3),2]}}\\
    \color{lightgray3}Return elements 1 and 3 from column 2\\
    \hline
    
    \texttt{\textbf{mat[,1]}}\\
    \color{lightgray3}Return all elements from column 1\\
    \hline
    
    \texttt{\textbf{mat[2,]}}\\
    \color{lightgray3}Return all elements from the second row\\
    \hline
    
    \texttt{\textbf{mat*10}}\\
    \color{lightgray3}Multiply all elements by 10\\
    \hline
  \end{tabularx}
  
  \section{\textbf{Import data from CSV and TXT}}
  \rowcolors{1}{lightgray1}{lightgray2}
  \begin{tabularx}{\textwidth/2}{X|l}  
     \texttt{\textbf{}}\\
    \\
    \hline
    
    \texttt{\textbf{}}\\
    \\
    \hline
    
     \texttt{\textbf{}}\\
    \\
    \hline
    
    \texttt{\textbf{}}\\
    \\
    \hline
    
     \texttt{\textbf{}}\\
    \\
    \hline
    
    \texttt{\textbf{}}\\
    \\
    \hline
    
     \texttt{\textbf{}}\\
    \\
    \hline
    
    \texttt{\textbf{}}\\
    \\
    \hline
    
    \texttt{\textbf{}}\\
    \\
    \hline
    
     \texttt{\textbf{}}\\
    \\
    \hline
    
    \texttt{\textbf{}}\\
    \\
    \hline
    
     \texttt{\textbf{}}\\
    \\
    \hline
    
    \texttt{\textbf{}}\\
    \\
    \hline
    
     \texttt{\textbf{}}\\
    \\
    \hline
    
    \texttt{\textbf{}}\\
    \\
    \hline
  \end{tabularx}

    
  \section{\textbf{Getting Started}}

  \newpage
  \pagebreak

  \end{multicols}
\end{document}